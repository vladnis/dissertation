\chapter{Conclusion and future work}
\label{chapter:conclusion}

We proposed a method to estimate the optimal number of participants to a crowd sensing campaign, on an area of interest, through the use of simulations. We used three scenarios for three different crowdsensing campaigns: air quality analysis, pedestrian density and WiFi access. The three crowd sensing campaigns were simulated over the same space of interest, Herastrau park, in order to show how the differences in the crowdsensing campaign, mainly the sensor type, affect the results. We extended our analysis by taking a look at how the duration of the campaign affects the results.

To show the relevance of the results we added a method to determine the number of people that have access to a crowdsensing area of interest, own a smartphone and may be willing to participate. We know that incentives can be used in order to increase the willingness of people to participate, but this is outside the scope of the article and can be considered future work.

As future work we need to offer more detailed simulations that better match specific crowdsensing campaigns. People do not normally move according to a random walk algorithm, but take specific paths, they also travel in groups and their behavior can be influenced by the campaign. The sensors can be modeled with higher precision, moving away from a disc radius. The simulation can take the environment into account as well as its effect on the sensors and the area that they cover. Finally, real life crowdsensing campaigns can be used to further improve the simulations.
