\chapter{Introduction}
\label{chapter:intro}

\section{Crowd Sensing}
\label{sec:intro-crowd}
There is a need for up to date, detailed data on different metrics that characterise the environments we live in. A few examples are air quality, noise pollution, traffic density and even WiFi access. This data is significant to environmental scientists, city managers and even citizens which can make better, informed decisions. Data is the first step in enabling smart cities.

Standard solutions for gathering environmental data on a large geographical scale are expensive and difficult to implement. Even community-driven projects such as the Air~Quality~Egg\footnote{\url{http://airqualityegg.com/}} come at a high price and have a moderate usage (in the order of thousands for the entire planet). A recent alternative that is growing in popularity is crowdsensing (Section \ref{sec:related-crowd}). Crowdsensing~\cite{riva2007urbanet} proposes the distribution of the data gathering task to a large number of people. Any person carrying a device capable of measuring a certain characteristic of the environment and transmitting this data to a central database can participate in a crowdsensing campaign.

Smartphones represent the most convenient tool for a crowd sensing campaign. First of all, smartphones are now ubiquitous, meaning the cost of deploying a crowdsensing campaign is lowered because most people already own the hardware needed for such a campaign. According to \cite{poushter2016smartphone}, in the United States, the penetration of smartphones has reached a percentage of 72\% and even in less developed countries the percentage of smartphone penetration is rising fast. They have a large variety of sensors, such as accelerometers, microphones and video cameras and can easily be extended with more through the use of Bluetooth. They have more and more powerful processors, which permit complex data processing, as well as clear methods of developing and deploying new applications that can run on a very large number of devices. Lastly, all smartphones have WiFi and LTE modules which permit the transmission of the sensed data.

For crowdsensing, to be effective, it still has to deal with a number of pitfalls. Crowdsensing campaigns require a significant number of participants in order to permit the extraction of valid conclusions (a.k.a., a critical mass of people is needed in order to validate the measurements). These campaigns are either volunteer-based or offer special incentives, such as monetary compensation \cite{ra2012medusa}. In either case, it is not clear how many participants are needed in order to have a successful crowdsensing campaign, where enough data is gathered to permit the organisers to draw valid conclusions. To our knowledge, there are no studies that offer a methodology for determining a threshold for an optimal number of participants of a crowdsensing campaign. The optimal number is the minimal number of participants for which enough data is gathered. The number of participants needs to be minimal in order to minimise resource usage and incentives costs.

In the first part of this study, we propose a methodology for computing the optimal number of participants for a crowdsensing campaign depending on the campaign characteristics. Having this number is vital for the organisers of any campaign that uses crowdsensing techniques. First of all, it offers an idea of the possible success (as in, how representative is the measurement data to describe a particular environment) of the campaign. Secondly, a relation between the number of participants and the effectiveness of the measurement sensing campaign, can, in fact, help the organisers plan the use of incentives to either motivate participation or keep the costs under control.

Determining an optimal number of participants in real life is difficult. Starting a crowd sensing application is currently based on the willingness of people to participate. It is unlikely and extremely time-consuming to repeat the same crowdsensing campaign with varying number of participants.

No trivial solutions for determining the optimal number of participants are available. An obvious one would be to divide the size of the area of interest, by the size of the sensing area of a single sensor. The solution is appropriate for static sensors. However, it does not take into account overlaps between multiple sensors or movements of the sensors, as they are carried around by people, during a time period. When we take movement into account a single sensor can cover a far larger area than it is expected from its specification.

Our approach for determining the optimal number of participants is to simulate crowdsensing campaigns that take into account crowd movement (Section \ref{sec:exp-crowd}). We choose an area of interest, Herastrau park in Bucharest, and simulate crowdsensing campaigns for air quality, WiFi access as well as people density as crowd sensing applications (Section \ref{sec:res-crowd}).

We take a further look into the problem of the number of participants by offering a discussion on the number of possible candidates for a crowd sensing campaign. Possible candidates represent people for which taking part in the campaign is reasonable. To offer a clear example, it is wrong to assume that a crowd sensing campaign could have more participants than the number of citizens of the respective city the campaign takes place in (Section \ref{sec:res-crowd-discution}).


\section{Inertial sensors}
\label{sec:intro-acc}

Inertial sensors, like accelerometers, gyroscopes, magnetic sensors or gravitational sensors, are one of the most used components,
because of the multitude of applications that can be built from
inertial information and also low costs and easy integration in different
types of hardware, like mobile phones, robots, drones and
many more.

This types of sensors can be used together in order to minimise the drift of the captured data. 
The role of an inertial sensor is to identify physical
movement, which can be linear displacement or rotation, and
transform it to a readable set of analogical or digital data.

An inertial sensor commonly ranges from low cost, MEMS inertial
sensors, measuring only a few square mm, that offer less
precision, up to more precise and expensive sensors, like ring
laser gyroscopes which can measure 50 cm in diameter.  Most
common types of sensors are MEMS ( Microelectromechanical ) inertial sensors, which can be found in most smartphones,
drones, head mount displays, IoT applications and others.
For example, a Nexus 5 smartphone uses InvenSense MPU-
6515 six-axis, which is a MEMS MotionTracking device and
includes capacitive gyros and accelerometers. The same IMU
is used in \cite{Liu-2015} for detecting unsafe driving. 

In an inertial system, the information from different sensors
are fused together to have the result that is expected. In
general, for fusing these types of information a Kalman filter
is used, which is able to combine data from several different
environments that have outputs with noise. The role of this
the filter is to use the combined data to reduce the weak point of
sensors, so that combining the best parts of different types of
sensors can result in better precision.

The most common inertial sensors are presented below:

\begin{itemize}
  \item When analysing an inertial sensor, one of the important components are accelerometers. The general idea of
functioning is based on a mass-spring system which
resides in a vacuum and is anchored to a fixed frame. The
acceleration sensing is done from the displacement of the
spring.

Depending on the construction mode, there are many
different categories that they fall into like based on the piezoresistive effect, that uses microscopic crystal structures or based on sensing changes in capacitance. Other types include thermal or optical accelerometers.

In terms of methods of integration, analog accelerometer
outputs a continuous voltage that is directly proportional
to the acceleration force, whereas digital accelerometers
use pulse width modulation. The amount of time
the voltage is up is proportional to the acceleration.
Analog accelerometers can be more precise and simpler
to integrate into a system, if analog inputs are supported.
The digital accelerometer has to use the timing resources
to measure the duty cycle, which implies running intense
computational division operations.
  \item In an inertial sensor, information is brought also from
a gyroscope.  In a MEMS inertial sensor, gyroscopes have a vibrating mass that is oscillating and using the Coriolis
effect, sense rotation. They have the advantage of being
small, low cost and low power. Other types would be mechanical, which work on the base of conservation of angular momentum or optical.
\item Magnetic sensors in MEMS inertial system are used to detect and measure the magnetic fields and uses the effects of Lorentz force to capture this types of data.
\end{itemize}

\section{Inertial sensors in crowd sensing applications}
\label{sec:intro-acc-crowd}

To give a more specific context to crowd sensing, we propose the use of inertial sensors to recognise when a user is using stairs or an elevator. Using data aggregated from a large number of smartphones, a map of stairs and elevators can be built, with minimum effort and costs. The beneficiaries for this map could be people with disabilities, that could route their path to a destination according to this information.

From this usecase, another one that relates to the one analysed is the use of stairway and elevator detecton to analyse, in an indoor environment, the level at which a specific user is. This use case would imply the measuring of each taken step or the distance that the elevator climbed, so that this distance is associated to an level in the building. 

For this use case, firstly, an Android application will be built for creating testing and training datasets. The training and testing data will be pre-segmented. If present, the magnetic and gravitational sensor will be used to reduce drift and normalise the captured data.

In the second step, a gesture recognition pipeline will be proposed, which will contain algorithms for pre-processing, so that the noise for the captured data is reduced, for gesture recognition, which will identify five actions: stairs up, stairs down, the beginning and the end of the elevator going up and walking. The pipeline can also contain post-processing algorithms, to correct possible classification errors.

The last step will be to test how fast the pipeline learns the proposed event and compare different algorithms results in terms of precision of classification. If the results imply that this use cases can be implemented, also it would be interesting to analyse the impact that the recognition pipeline has in the cotext of using a smartphone and if it can be integrated in such a device for continious data stream classification. Also, an important problem is the data segmentation from a continous stream of data.

