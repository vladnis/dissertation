\chapter{Related Work}
\label{chapter:related}


\section{Crowd Sensing}
\label{sec:related-crowd}
Crowdsensing is still a young scientific area but it has already gathered a lot of interest and support, as we show below. There are a great number of studies that show the diversity of scenarios in which crowdsensing can be applied.

As stated in \cite{lane2010survey} and \cite{khan2013mobile}, crowdsensing can provide micro and macroscopic analysis of cities, communities and persons. It can be applied in social networking, health, energy, monitoring human behavior and many others.

In terms of environmental analysis, the authors of \cite{guo2014participatory} show that by using crowdsensing in urban areas one can collect traffic data or generate noise maps. A similar solution, regarding noise pollution in urban areas is presented in \cite{stevens2010crowdsourcing}. Air quality analysis represents another interesting use for crowdsensing. Authors in \cite{zheng2013u} claim that crowdsensing is the optimal solution for capturing data for large areas, although it is only feasible for \(CO_2\) emissions, because of the complexity and cost of other sensors. \cite{ma2014opportunities} adds road surface monitoring and street parking availability. The former article also presents two categories of sensing: participatory, in which users have to get involved and be active in the measurement and data acquisition process, and opportunistic, where the user has a more limited role in the sensing. In fact, the second sensing approach tends to also be more visible in the literature and existing platforms for crowdsensing, because the user is relieved of much of the burdens associated with taking decisions where and how to collect measurement data. Our work focuses on the second approach, opportunistic sensing.

Without the use of crowdsensing the cost of implementing a similarly capable sensor system would be significantly higher. In \cite{mao2012citysee} the authors show that to cover an area of approximatively $1km^2$, 100 sensors and 1096 relays need to be used. This method cannot scale for larger area of studies, so crowd sensing is a better alternative.

As described in \cite{ra2012medusa}, crowd sensing generally implies a requester, a campaign starter, which requires users willing to capture sensor data that is used directly or at a later time for various experiments. The authors acknowledge that in terms of user engagement, it is sometimes difficult to convince users to participate in a crowdsensing analysis. Incentives are proposed as a solution to this problem. They take the form of payments or gamification. Incentives are also used in \cite{talasila2014crowdsensing}. The authors use micro-payments and gamification to attract enough users so that they can cover the area of a university campus. The scope was to measure WiFi signal strength across the entire campus. However, the authors make no analysis on what is the optimal number of participants to a crowd sensing campaign or how to manage incentives to reach that value. The use of payments for crowd sensing participation without a previous analysis of possible number of participants could lead to high costs and little control over the results.

\textbf{Access to a space of interest}.
It is not enough to determine the optimal number of participants for a crowd sensing campaign. One also needs to take into account the plausibility of achieving this number. Probably one of the factors that is most relevant is the accessibility of individuals to the area of interest. In the literature there are different approaches to determine the number of people that have access to an area, such as a park, some of which are presented below.

The authors of \cite{nicholls2001measuring} present the travel cost approach. They show two methods of measuring the accessibility to a park using geographical information. In the first, the service area of the park is considered a circle, with the radius equal to the maximum desired distance between the center of the park and locations of visitors. The disadvantage of this method is that park visitors are assumed to be able to reach the park by using a straight line, which is not usually the case. The second method is based on the shortest path algorithm using the boundary of the park area and the streets and paths that the visitors can walk on. Their work takes into account both neighborhood parks (2 - 4 ha) and mini neighborhood and community parks (0.4 - 2 ha). The total distance accepted is less than 0.8 km of walking.

The main problem with this approach is that visitors do not always use the closest park and tend to travel more for a park that has a bigger area or more facilities compared to the ones offered by a closer park. An alternative would be consider the distance cost to all parks in the studied area, or a predefined set. Using this method, a set of 3 parks and limited number of people it has been demonstrated \cite{iamtrakul2005public} that visitors prefer parks that offer more attractiveness even if they are not in close proximity.

The authors of \cite{talen1998assessing} use the container approach to determine the equity and accessibility to public playground. An extension of this method is the kernel algorithm presented in \cite{maroko2009complexities}. The Kernel density estimation method can give an accessibility rating for every point in the studied area. The idea behind this method is to rate every acre that is studied as belonging to a park or not and using kernel density algorithm, the area is converted to a statistical surface. The problem with this method is that depending on the kernel bandwidth taken into consideration, there could be areas that have no accessibility, which is incorrect.

The attractiveness of a park can be modeled by using as little as only the size of a park. The assumption is that larger parks have more facilities and respectively are more attractive. In \cite{zhang2011modeling} the authors propose the population-weighted distance, which should be a more precise method of determining the accessibility to a park. This is the method we chose for our analysis and discussion.

\section{Accelerometers}
\label{sec:related-acc}
