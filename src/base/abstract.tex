% This file contains the abstract of the thesis

Having access to up-to-date and accurate information about the environment can 
be very important in decision making for both an environmental scientist, city planners, 
park administrators and also regular users. One of the easiest ways of capturing this type of data is through inertial sensors.

 Inertial sensors have many use cases in the context of gathering environmental data, one of them being capturing
surface movements like earthquakes or vibrations caused by different 
factors, while other can include gesture recognition, user behaviour
in specific spaces or analysing the terrain that the user is going through.

Even more powerful applications can be built from gathering inertial data from
a large number of devices, which can be achieved
 through crowd sensing. Other alternatives for acquiring such data by hand
or by deploying sensor networks would be expensive and time-consuming.
Also, people without proper scientific
training want to take part in the data acquisition process and assist
with the needs of their respective communities.


Crowdsensing is a
technology-driven area where ICT platforms are being developed
which permit anyone to participate in processes that help expand
our understanding and improve our surroundings. Crowdsensing
refers to the process in which crowds (large number of people)
measure specific features and share the resulting data or send it
to a central location in which it can be used by the people that
need it. Since the
increase in popularity of smartphones, which are now ubiquitous,
crowdsensing is more popular than ever.

A lot of people can now participate to crowdsensing but we still
don’t know how many are needed for a crowdsensing campaign
to be successful. In order to answer this question, we built a
simulator that mimics the characteristics of a crowdsensing campaign.
 We showcase three different scenarios in which we estimate
the required number of participants and offer a discussion on
the plausibility of having that many participants by taking into
account factors such as accessibility to the area of interest (the
area from which the measurements are needed).

Also, using crowd sensing, inertial sensors and gesture recognition, a use case is proposed and analysed in more
 detail. We propose a way to gather information about the location
 of stairs and elevators, that could help disabled
 persons calculate paths to different destinations.


