% This file contains the abstract of the thesis

Having access to up-to-date and accurate information about the environment can 
be important for both environmental scientist, city planners, 
park administrators and also regular users. One way of capturing this type of data is through inertial sensors. An example of application for inertial data can be
surface movements like earthquakes or vibrations caused by diferrent 
factors, while other can include gesture recognition, user behavior
in specific spaces or analysing the terrain that the user is going through.

Powerfull applications can be built from gathering inertial data from
a large number of devices with inertial sensors. This can be achived
 through crowd sensing. Other alternatives for acquiring such data by hand
or by deploying sensor networks would be expensive and time consuming.
Also, people without proper scientific
training want to take part in the data acquisition process and assist
with the needs of their respective communities.


Crowdsensing is a
technology-driven area where ICT platforms are being developed
which permit anyone to participate in processes that help expand
our understanding and improve our surroundings. Crowdsensing
refers to the process in which crowds (large number of people)
measure specific features and share the resulting data or send it
to a central location in which it can be used by the people that
need it. Since the
increase in popularity of smartphones, which are now ubiquitous,
crowdsensing is more popular than ever.

A lot of people can now participate to crowdsensing but we still
don’t know how many are needed for a crowdsensing campaign
to be successful. In order to answer this question, we built a
simulator that mimics the characteristics of a crowdsensing campaign.
 We showcase three different scenarios in which we estimate
the required number of participants and offer a discussion on
the plausibility of having that many participants by taking into
account factors such as accessibility to the area of interest (the
area from which the measurements are needed).

Also, using crowd sensing, a use case is proposed and analysed in more
 detail. We propose a way to gather information about location
 of stairs and elevators, using gesture recognition, that could help disabeled
 persons calculate paths to diferent destinations.

